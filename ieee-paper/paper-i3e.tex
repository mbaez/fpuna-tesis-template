
\documentclass[conference]{IEEEtran}

\usepackage[style=numeric, uniquename=full, sorting=none,backend=biber, natbib=true]{biblatex}

%se importan las configuraciones custmizdas realizadas.
\addbibresource{../book/referencias.bib}

\usepackage[spanish]{babel}
\usepackage[utf8]{inputenc}

% *** GRAPHICS RELATED PACKAGES ***
%
\usepackage[pdftex]{graphicx}


% *** MATH PACKAGES ***
%
\usepackage{amsmath,amsthm}
\usepackage{amssymb}
\usepackage{textcomp}

% *** SPECIALIZED LIST PACKAGES ***
\usepackage{algorithm}
%\usepackage{algorithmic}
\usepackage{algpseudocode}


%Traducción al español del paquete algorithmic%
\floatname{algorithm}{Algoritmo}
\renewcommand{\algorithmicrequire}{\textbf{Entrada:}}
\renewcommand{\algorithmicensure}{\textbf{Salida:}}
\renewcommand{\algorithmicend}{\textbf{fin}}
\renewcommand{\algorithmicif}{\textbf{si}}
\renewcommand{\algorithmicthen}{\textbf{entonces}}
\renewcommand{\algorithmicelse}{\textbf{si no}}
\newcommand{\algorithmicelsif}{\textbf{si no, si}}
\renewcommand{\algorithmicfor}{\textbf{para}}
\renewcommand{\algorithmicforall}{\textbf{para todo}}
\renewcommand{\algorithmicdo}{\textbf{hacer}}
 %\renewcommand{\algorithmicendfor}{\algorithmicend\ \algorithmicfor}
% \renewcommand{\algorithmicwhile}{\textbf{mientras}}
% \renewcommand{\algorithmicendwhile}{\algorithmicend\ \algorithmicwhile}
% \renewcommand{\algorithmicloop}{\textbf{repetir}}
% \renewcommand{\algorithmicendloop}{\algorithmicend\ \algorithmicloop}
% \renewcommand{\algorithmicrepeat}{\textbf{repetir}}
% \renewcommand{\algorithmicuntil}{\textbf{hasta que}}
% \renewcommand{\algorithmicprint}{\textbf{imprimir}}
\renewcommand{\algorithmicreturn}{\textbf{retorna}}
% \renewcommand{\algorithmictrue}{\textbf{cierto }}
% \renewcommand{\algorithmicfalse}{\textbf{falso }}
% *** ALIGNMENT PACKAGES ***
%
%\usepackage{array}
%\usepackage{mdwmath}
%\usepackage{mdwtab}

%\usepackage{eqparbox}
\usepackage{threeparttable}
\usepackage{caption}
\usepackage{subcaption}
%\usepackage[caption=false]{caption}
%\usepackage[font=footnotesize]{subfig}

%\usepackage[caption=false,font=footnotesize]{subfig}

% *** FLOAT PACKAGES ***
%
%\usepackage{fixltx2e}
%\usepackage{stfloats}

\newcommand{\figref}[1]{Figura \ref{#1}}
\newcommand{\tabref}[1]{Cuadro \ref{#1}}
\newcommand{\secref}[1]{sección \ref{#1}}
\renewcommand{\algref}[1]{Algoritmo \ref{#1}}

% correct bad hyphenation here
\hyphenation{}


\begin{document}
%
% paper title
% can use linebreaks \\ within to get better formatting as desired
\title{Modelo predictivo de focos de dengue aplicado a Sistemas de Información Geográfica}


% author names and affiliations
% use a multiple column layout for up to three different
% affiliations
\author{\IEEEauthorblockN{Autor del trabjao \IEEEauthorrefmark{1},
Tutor del trabajo\IEEEauthorrefmark{2}}
\IEEEauthorblockA{Facultad Politécnica- Universidad Nacional de Asunción\\
    P.O.Box: 2111 SL, San Lorenzo - Central - Paraguay\\
Email: \IEEEauthorrefmark{1}autor@mail.com,
\IEEEauthorrefmark{2}tutor@mail.com}}

% make the title area
\maketitle

\begin{abstract}
%\boldmath
Abstrac del trabjo
\end{abstract}

\begin{IEEEkeywords}
test, tempalte, IEEE
\end{IEEEkeywords}



% For peerreview papers, this IEEEtran command inserts a page break and
% creates the second title. It will be ignored for other modes.
\IEEEpeerreviewmaketitle

%introduccion

\section{Introduction}
This demo file is intended to serve as a ``starter file''
for IEEE conference papers produced under \LaTeX\ using
IEEEtran.cls version 1.7 and later.

I wish you the best of success.

\subsection{Subsection Heading Here}
Subsection text here.

\subsubsection{Subsubsection Heading Here}
Subsubsection text here.

\section{Figuras, tablas y algoritmos}
\label{sec:figuras-tablas-algoritmos}
\subsection{Figuras}
Ejemplo de como hacer figuras utilizando el paquete figure\footnote{http://en.wikibooks.org/wiki/LaTeX/Floats,\_Figures\_and\_Captions} de LaTex.
\begin{figure}[H]
    \centering
    \begin{subfigure}[b]{0.225\textwidth}
            \includegraphics[width=\textwidth]{../book/capitulo-ej/graphics/ejemplo-1.jpg}
            \caption{Subfigura 1.}
    \end{subfigure}
    ~~~~
    \begin{subfigure}[b]{0.225\textwidth}
            \includegraphics[width=\textwidth]{../book/capitulo-ej/graphics/ejemplo-1.jpg}
            \caption{Subfigura 2.}

    \end{subfigure}
    \begin{subfigure}[b]{0.225\textwidth}
            \includegraphics[width=\textwidth]{../book/capitulo-ej/graphics/ejemplo-1.jpg}
            \caption{Subfigura 3.}
    \end{subfigure}
    ~~~~
    \begin{subfigure}[b]{0.225\textwidth}
            \includegraphics[width=\textwidth]{../book/capitulo-ej/graphics/ejemplo-1.jpg}
            \caption{Subfigura 4.}

    \end{subfigure}
    \caption{\label{fig:ejemplo-fig-grilla}Ejemplo de una grilla de figuras en LaTex.}

\end{figure}

\section{Tablas}
Ejemplo de como hacer tablas utilizando el paquete table\footnote{http://en.wikibooks.org/wiki/LaTeX/Tables} de LaTex.
\begin{table}[!hptb]
\begin{threeparttable}
    \begin{minipage}[b]{0.5\textwidth}
        \caption{\label{tab:tabla-ejemplo} Ejemplo de una tabla en LaTex.}
         \footnotesize
        \begin{tabular}{l c c c c}
            \hline\\
            Año & Periodo & Columna & Columna2 & Columna3\\
            \hline
            \hline\\
            2014 & 29-12-13 / 31-05-14 & 10541 & 1052 & 2\\
            2013 & 30-12-12 / 21-12-13 & 153793 & 131306 & 70\\
            2012 & 01-01-12 / 22-12-12 & 37815 & 30588 & 11\\
            2011 & 03-01-11 / 29-12-11 & 53397 & 42264 & 62\\
            2010 & 11-10-09 / 25-12-10 & 21951 & 13760 & --\tnote{a}\\
        \hline
    \end{tabular}
    \begin{tablenotes}[flushleft]\footnotesize
    \item[a]{Esta es una nota.}
    \end{tablenotes}
    \end{minipage}
    \end{threeparttable}
\end{table}


\subsection{Algoritmos}
Ejemplo de como hacer agloritmos utilizando el paquete algorithm\footnote{http://en.wikibooks.org/wiki/LaTeX/Algorithms} de LaTex.
\begin{algorithm}                      % enter the algorithm environment
\caption{\label{alg:alg1}Calculate $y = x^n$}          % give the algorithm a caption
\begin{algorithmic}[1]                 % enter the algorithmic environment
    \Require $n \geq 0 \vee x \neq 0$
    \Ensure $y = x^n$
    \State $y \Leftarrow 1$
    \If{$n < 0$}
        \State $X \Leftarrow 1 / x$
        \State $N \Leftarrow -n$
    \Else
        \State $X \Leftarrow x$
        \State $N \Leftarrow n$
    \EndIf
    \While{$N \neq 0$}
        \If{$N$ is even}
            \State $X \Leftarrow X \times X$
            \State $N \Leftarrow N / 2$
        \Else[$N$ is odd]
            \State $y \Leftarrow y \times X$
            \State $N \Leftarrow N - 1$
        \EndIf
    \EndWhile
\end{algorithmic}
\end{algorithm}

\subsection{Ecuaciones}
En esta sección se añade una pequeña ecuación con el fin de ejemplificar su
uso.
\begin{equation}\label{eq:ecuacion-ej}
  x = a_0 + \cfrac{1}{a_1
          + \cfrac{1}{a_2
          + \cfrac{1}{a_3 + \cfrac{1}{a_4} } } }
\end{equation}

\section{Referencias y citaciones}
Para referenciar secciones, figuras, tablas, algoritmos, o formulas se puede
emplear $\setminus$ref\{label-del-item\} o emplear cualquiera de las sigientes macros:

\begin{itemize}
\item \textit{$\setminus$eqref\{label-eq\}} : Ejemplo \eqref{eq:ecuacion-ej}
\item \textit{$\setminus$secref\{label-sec\}} : Ejemplo \secref{sec:figuras-tablas-algoritmos}
\item \textit{$\setminus$tabref\{label-tab\}} : Ejemplo \tabref{tab:tabla-ejemplo}
\item \textit{$\setminus$figref\{label-fig\}} : Ejemplo \figref{fig:ejemplo-fig-grilla}
\item \textit{$\setminus$algref\{label-alg\}} : Ejemplo \algref{alg:alg1}
\end{itemize}

En esta sección se añade un ejemplo de como citar a un autor, utilizando
el $\setminus$cite\{label-bib1\} de bibText \footnote{http://en.wikibooks.org/wiki/LaTeX/Bibliography\_Management}.
Por ejemplo esta es una citación \cite{griffiths1997learning} a un solo
autor, y esta es a 2 autores \cite{griffiths1997learning, lamport1985i1}.


\section{Conclusion}
The conclusion goes here.

% use section* for acknowledgement
\section*{Acknowledgment}
The authors would like to thank...


%Materiales y metodos
%\input{materiales-metodos}
%Resultados y discusión
%\input{resultados-discusion}
%conclusión final
%\input{conclusion}

\printbibliography

% that's all folks
\end{document}


