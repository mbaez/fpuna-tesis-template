%%%
%%% Conjuto de funciones utilitarias
%%% autor: Maximiliano Báez
%%% fecha: 25/08/2014
%%%

\usepackage{tablefootnote} %para utilizar \footnote{}
\usepackage{amssymb}
\renewcommand{\thefootnote}{\arabic{footnote}}

%%Para la construcción de la lista de simbolos
% Macro for 'List of Symbols', 'List of Notations' etc...
\def\listofsymbols{
    \newpage
\chapter*{Lista de Símbolos\hfill}
\addcontentsline{toc}{chapter}{Lista de Símbolos}
\begin{tabbing}
% YOU NEED TO ADD THE FIRST ONE MANUALLY TO ADJUST THE TABBING AND SPACES
$n$~~~~~~~~~~\=\parbox{5in}{Vector size\dotfill \pageref{symbol:nml}}\\
%ADD THE REST OF SYMBOLS WITH THE HELP OF MACRO

%% se añaden nuevos simbolos con el macro \newsymbol y se hace referecnia
% al simbolo utilizando \addsymbol{symbol:LABEL}

%\newsymbol (x_i, y_i): {coordenadas que representan un punto}{symbol:xy_i}

\end{tabbing}

    \clearpage{}
}
\def\newsymbol #1: #2#3{$#1$ \> \parbox{5in}{#2 \dotfill \pageref{#3}}\\}
\def\addsymbol#1{\label{#1}}

% Para las imagenes en grilla
% custom commands
\newcommand{\foreign}[1]{{\it #1}}
\DeclareMathOperator*{\argmax}{arg\,max}
%\algsetup{}
\algsetup{
    indent=4em,
    linenosize=\small,
    linenodelimiter=.
}

\usepackage{amsmath}

%% se utilizan para referenciar figuras, tablas, secciones y algoritmos
\newcommand{\figref}[1]{Figura \ref{#1}}
\newcommand{\tabref}[1]{Tabla \ref{#1}}
\newcommand{\secref}[1]{sección \ref{#1}}
\newcommand{\algref}[1]{Algoritmo \ref{#1}}


%Traducción al español del paquete algorithmic%
\floatname{algorithm}{Algoritmo}
\renewcommand{\listalgorithmname}{Lista de algoritmos}
\renewcommand{\algorithmicrequire}{\textbf{Entrada:}}
\renewcommand{\algorithmicensure}{\textbf{Salida:}}
\renewcommand{\algorithmicend}{\textbf{fin}}
\renewcommand{\algorithmicif}{\textbf{si}}
\renewcommand{\algorithmicthen}{\textbf{entonces}}
\renewcommand{\algorithmicelse}{\textbf{si no}}
\renewcommand{\algorithmicelsif}{\algorithmicelse,\ \algorithmicif}
\renewcommand{\algorithmicendif}{\algorithmicend\ \algorithmicif}
\renewcommand{\algorithmicfor}{\textbf{para}}
\renewcommand{\algorithmicforall}{\textbf{para todo}}
\renewcommand{\algorithmicdo}{\textbf{hacer}}
\renewcommand{\algorithmicendfor}{\algorithmicend\ \algorithmicfor}
\renewcommand{\algorithmicwhile}{\textbf{mientras}}
\renewcommand{\algorithmicendwhile}{\algorithmicend\ \algorithmicwhile}
\renewcommand{\algorithmicloop}{\textbf{repetir}}
\renewcommand{\algorithmicendloop}{\algorithmicend\ \algorithmicloop}
\renewcommand{\algorithmicrepeat}{\textbf{repetir}}
\renewcommand{\algorithmicuntil}{\textbf{hasta que}}
\renewcommand{\algorithmicprint}{\textbf{imprimir}}
\renewcommand{\algorithmicreturn}{\textbf{retorna}}
\renewcommand{\algorithmictrue}{\textbf{cierto }}
\renewcommand{\algorithmicfalse}{\textbf{falso }}
\renewcommand{\algorithmiccomment}{\textbf{comentario : }}
