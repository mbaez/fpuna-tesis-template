\documentclass[oneside]{beamer}


%*****************************************************************************%
%   Theme para la Presentación                                                %
%*****************************************************************************%
\usetheme[pageofpages=de,% String used between the current page and thetotal page count.
          bullet=circle,% Use circles instead of squares for bullets.
          titleline=true,% Show a line below the frame title.
          alternativetitlepage=true,% Use the fancy title page.
          titlepagelogo=../book/graphics/logo.png,% Logo for the first page.
          ]{Torino}
% Nouvelle is a green and red alternative to the chameleon color theme.
\usecolortheme{freewilly}
\providecommand*{\eqcolon}{\mathrel{=\!\raise.16ex\hbox{\footnotesize\!:}}} % para las definiciones

%*****************************************************************************%
%   Paquetes principales                                                      %
%*****************************************************************************%
\usepackage{amssymb, amsmath}
\usepackage{amsthm}
\usepackage{amsfonts}
\usepackage{float}
\usepackage{afterpage}
\usepackage{dsfont}
\usepackage{url}
\usepackage{color}
\usepackage{lettrine}
\usepackage{algorithm}
\usepackage{algpseudocode}
\usepackage{graphicx}
%\usepackage{subfig}
\usepackage{caption}
\usepackage{subcaption}
\usepackage{multimedia}
\usepackage{verbatim} % comentarios
\usepackage{pdfpages}
\usepackage[spanish]{babel}
\usepackage[utf8]{inputenc}
\usepackage{textcomp}

%Para determinar la fecha actual
\newcommand{\monthname}{\ifcase\month\or Enero\or Febrero\or
      Marzo\or Abril\or Mayo\or Junio\or Julio\or Agosto\or Septiembre\or
      Octubre\or Noviembre\or Diciembre\fi}

\newcommand{\thismonth}{\monthname,\ \the\year}


\author{\  \\ Autor del trabajo}
\title{Titulo del trabajo}
\institute{Facultad Politécnica- UNA}
%pone la fecha de generación como portada
\date{\thismonth}


\begin{document}
%para forzar el total de número de páginas, se utiliza para no tener en cuenta las páginas del anexo
%\renewcommand{\inserttotalframenumber}{71}
\begin{frame}[t,plain]
\titlepage
\end{frame}

\section{Introduction}
This demo file is intended to serve as a ``starter file''
for IEEE conference papers produced under \LaTeX\ using
IEEEtran.cls version 1.7 and later.

I wish you the best of success.

\subsection{Subsection Heading Here}
Subsection text here.

\subsubsection{Subsubsection Heading Here}
Subsubsection text here.

\section{Figuras, tablas y algoritmos}
\label{sec:figuras-tablas-algoritmos}
\subsection{Figuras}
Ejemplo de como hacer figuras utilizando el paquete figure\footnote{http://en.wikibooks.org/wiki/LaTeX/Floats,\_Figures\_and\_Captions} de LaTex.
\begin{figure}[H]
    \centering
    \begin{subfigure}[b]{0.225\textwidth}
            \includegraphics[width=\textwidth]{../book/capitulo-ej/graphics/ejemplo-1.jpg}
            \caption{Subfigura 1.}
    \end{subfigure}
    ~~~~
    \begin{subfigure}[b]{0.225\textwidth}
            \includegraphics[width=\textwidth]{../book/capitulo-ej/graphics/ejemplo-1.jpg}
            \caption{Subfigura 2.}

    \end{subfigure}
    \begin{subfigure}[b]{0.225\textwidth}
            \includegraphics[width=\textwidth]{../book/capitulo-ej/graphics/ejemplo-1.jpg}
            \caption{Subfigura 3.}
    \end{subfigure}
    ~~~~
    \begin{subfigure}[b]{0.225\textwidth}
            \includegraphics[width=\textwidth]{../book/capitulo-ej/graphics/ejemplo-1.jpg}
            \caption{Subfigura 4.}

    \end{subfigure}
    \caption{\label{fig:ejemplo-fig-grilla}Ejemplo de una grilla de figuras en LaTex.}

\end{figure}

\section{Tablas}
Ejemplo de como hacer tablas utilizando el paquete table\footnote{http://en.wikibooks.org/wiki/LaTeX/Tables} de LaTex.
\begin{table}[!hptb]
\begin{threeparttable}
    \begin{minipage}[b]{0.5\textwidth}
        \caption{\label{tab:tabla-ejemplo} Ejemplo de una tabla en LaTex.}
         \footnotesize
        \begin{tabular}{l c c c c}
            \hline\\
            Año & Periodo & Columna & Columna2 & Columna3\\
            \hline
            \hline\\
            2014 & 29-12-13 / 31-05-14 & 10541 & 1052 & 2\\
            2013 & 30-12-12 / 21-12-13 & 153793 & 131306 & 70\\
            2012 & 01-01-12 / 22-12-12 & 37815 & 30588 & 11\\
            2011 & 03-01-11 / 29-12-11 & 53397 & 42264 & 62\\
            2010 & 11-10-09 / 25-12-10 & 21951 & 13760 & --\tnote{a}\\
        \hline
    \end{tabular}
    \begin{tablenotes}[flushleft]\footnotesize
    \item[a]{Esta es una nota.}
    \end{tablenotes}
    \end{minipage}
    \end{threeparttable}
\end{table}


\subsection{Algoritmos}
Ejemplo de como hacer agloritmos utilizando el paquete algorithm\footnote{http://en.wikibooks.org/wiki/LaTeX/Algorithms} de LaTex.
\begin{algorithm}                      % enter the algorithm environment
\caption{\label{alg:alg1}Calculate $y = x^n$}          % give the algorithm a caption
\begin{algorithmic}[1]                 % enter the algorithmic environment
    \Require $n \geq 0 \vee x \neq 0$
    \Ensure $y = x^n$
    \State $y \Leftarrow 1$
    \If{$n < 0$}
        \State $X \Leftarrow 1 / x$
        \State $N \Leftarrow -n$
    \Else
        \State $X \Leftarrow x$
        \State $N \Leftarrow n$
    \EndIf
    \While{$N \neq 0$}
        \If{$N$ is even}
            \State $X \Leftarrow X \times X$
            \State $N \Leftarrow N / 2$
        \Else[$N$ is odd]
            \State $y \Leftarrow y \times X$
            \State $N \Leftarrow N - 1$
        \EndIf
    \EndWhile
\end{algorithmic}
\end{algorithm}

\subsection{Ecuaciones}
En esta sección se añade una pequeña ecuación con el fin de ejemplificar su
uso.
\begin{equation}\label{eq:ecuacion-ej}
  x = a_0 + \cfrac{1}{a_1
          + \cfrac{1}{a_2
          + \cfrac{1}{a_3 + \cfrac{1}{a_4} } } }
\end{equation}

\section{Referencias y citaciones}
Para referenciar secciones, figuras, tablas, algoritmos, o formulas se puede
emplear $\setminus$ref\{label-del-item\} o emplear cualquiera de las sigientes macros:

\begin{itemize}
\item \textit{$\setminus$eqref\{label-eq\}} : Ejemplo \eqref{eq:ecuacion-ej}
\item \textit{$\setminus$secref\{label-sec\}} : Ejemplo \secref{sec:figuras-tablas-algoritmos}
\item \textit{$\setminus$tabref\{label-tab\}} : Ejemplo \tabref{tab:tabla-ejemplo}
\item \textit{$\setminus$figref\{label-fig\}} : Ejemplo \figref{fig:ejemplo-fig-grilla}
\item \textit{$\setminus$algref\{label-alg\}} : Ejemplo \algref{alg:alg1}
\end{itemize}

En esta sección se añade un ejemplo de como citar a un autor, utilizando
el $\setminus$cite\{label-bib1\} de bibText \footnote{http://en.wikibooks.org/wiki/LaTeX/Bibliography\_Management}.
Por ejemplo esta es una citación \cite{griffiths1997learning} a un solo
autor, y esta es a 2 autores \cite{griffiths1997learning, lamport1985i1}.


\section{Conclusion}
The conclusion goes here.

% use section* for acknowledgement
\section*{Acknowledgment}
The authors would like to thank...


%~ \include{./propuesta/objetivos}
%~ \include{./propuesta/modelo-propuesto}
%~ \include{./resultados/resultados}
%~ \include{./conclusion/conclusion}
%~ \include{./conclusion/trabajos-futuros}
%~ \include{./conclusion/aportes}
%~ \include{./conclusion/despedida}

%\include{./anexos/anexos}

\end{document}
